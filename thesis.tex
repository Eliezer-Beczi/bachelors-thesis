% BEÁLLÍTÁSOK - JOBB NEM VÁLTOZTATNI
\documentclass[final]{ubb_dolgozat}
\usepackage{definitions}


% milyen nyelveken akarunk forráskódot megjeleníteni
\lstloadlanguages{Python}
% más lehetőségek:
% C, Matlab, Mathematica, Octave, Pascal, Perl, Python
% SCilab, SQL, Haskell, Lisp, Lua, make, ML, PHP, Prolog
%
% a teljes lista a LISTINGS csomagban.


% ezt be lehet tenni MINDEGYIK megjelenítendő kód elé opcióként
\lstset{language=Python}


%%%%%%%%%%%%%%%%%%%%%%%%%%%%%%%%%%%%%%%%%%%%%%%
%%!!          EZT KELL VÁLTOZTATNI       !!%%%%
%%     A DOLGOZAT CÍMOLDALÁNAK ELEMEI        %%

%% MELYIK ÉVBEN ADJUK LE
\submityear{%
2020
}

\titleHU{%
Kritikus csomópontok meghatározása komplex hálózatokban
}

% Az alábbi sorokat ki kell tölteni!!

\titleEN{%
Critical node detection problem in complex networks
}

\titleRO{%
Titlu lucrare licență
}

\author{%
Béczi Eliézer
}

%%
\tutorHU{%
dr. Gaskó Noémi,\newline egyetemi adjunktus\\
% a hozzátartozás akkor szükséges, ha NEM BBTE-s a tanár
%{\large Babe\c{s}--Bolyai Tudományegyetem,\\
% Matematika és Informatika Kar}% ha különbözik, akkor fel kell tűntetni
}
%%
\tutorRO{%
Lector dr. Gaskó Noémi\\
% az egyetem akkor szükséges, ha nem BBTE-s a tanár, a minta a BBTE-t
% tartalmazza
% {\large Universitatea Babe\c{s}--Bolyai,\\ % dacã diferã!!!
% Facultatea de Matematic\u{a} \c{s}i Informatic\u{a} }%
}
%%
\tutorEN{%
Assist prof. dr. Gaskó Noémi
% {\large Babe\c{s}--Bolyai University,\\
% Faculty of Mathematics and Informatics}
}


% 
%\includeonly{bevezet}


\begin{document}

%% ABSTRAKT
\begin{abstractEN} % ANGOL VÁLTOZAT

  % a lenti részt értelemszerűen ki kell tölteni a dolgozat angol kivonatával.
  % A BEGIN ... END között CSAK A SAJÁT SZÖVEG kell, hogy legyen.
  % Az utolsó mondatot benne kell hagyni, mely által kijelentitek, hogy a munkátok SAJÁT.


  {

    \vfill

    \center{

      {\huge EZ AZ OLDAL NEM RÉSZE A DOLGOZATNAK!}

      \vspace{0.5cm}

      {\Large Ezt az angol kivonatot külön lapra kell nyomtatni és alá kell írni!}

      \vspace{0.5cm}

      {\huge A DOLGOZATTAL EGYÜTT KELL BEADNI!}

    }

    \vfill

    Kötelező befejezés:
  }
  \vspace*{.5cm}

  This work is the result of my own activity. I have neither given nor received unauthorized assistance on this work.

\end{abstractEN}

% ez a címoldal része
\maketitle

%% 

% a dolgozat tartalomjegyzéke -- ez automatikusan generálódik a STRUKTÚRA alapján.
{ \baselineskip 1ex
  \parskip 1ex
  \tableofcontents
}


%%%%%%%%%%%%%%%%%%%%%%%%%%%%%%%%%%%%%%%%%%%%%%%%%%%%%%%%%%%
%%%%%%%%%%         a dolgozat tartalma         %%%%%%%%%%%%

% ajánlott külön file-okba írni az egyes fejezeteket,
% ugyanis úgy jobban át lehet látni.


% a bevezető fejezet FILE-ja.
\include{bevezet}

% !TeX root = ../../thesis.tex
%%%%%%%%%%%%%%%%%%%%%%%%%%%%%%%%%%%%%%%%%%%%%%%%%%%%%%%%%%%%%%%%%%%%%%%
\chapter{Bevezető}\label{ch:BEVEZETO}

Hálózatok terén nem minden csomópont egyforma fontosságú.
A kulcsfontosságú csomópontok keresésével hálózatokban széles körben foglalkoznak,
különösképpen olyan csomópontok esetén, melyek a hálózat konnektivitásához köthetők.
Ezeket a csomópontokat általában úgy nevezzük, hogy Kritikus Csomópontok.

Kritikus Csomópontok Meghatározásának Problémája (CNDP)
egy optimalizációs feladat, amely egy olyan csoport csomópont
megkereséséből áll, melyek törlése maximálisan rontja a hálózat
konnektivitását bizonyos predefiniált konnektivitási metrikák szerint.

A CNDP számos alkalmazási területtel rendelkezik.
Például, közösségi hálók nagy befolyással bíró egyedeinek azonosítása,
komputációs biológiában kapcsolatok definiálására jelút
vagy fehérje-fehérje kölcsönhatás hálózatokban,
smart grid sebezhetőségének azonosítása, egyének meghatározása
védőoltással való ellátásra vagy karanténba való zárásra egy
fertőzés terjedésének gátlása érdekében.

A CNDP egy $\mathcal{N}\mathcal{P}$-teljes feladat. Adva van egy $G = (V, E)$ gráf, ahol $|V| = n$ a csomópontok száma,
és $|E| = m$ pedig az élek száma. A feladat $K$ kritikus csomópont meghatározása, amelyek törlése a bemeneti
gráfból minimalizálja a hálózat páronkénti konnektivitását. Az alapján, hogy mit értünk egy hálózat
konnektivitása alatt, a CNDP-nak van egycélú illetve többcélú megfogalmazása is.

Ebben a dolgozatban többek között egy bi-objektív megfogalmazásával fogunk foglalkozni a CNDP-nak.
Hat standard evolúciós algoritmust (NSGAII, EpsMOEA, SPEA2, IBEA, PAES, EpsNSGAII) fogunk összehasonlítani
egymással különböző szintetikus bemenetekre (Barabási-Albert, Watts-Strogatz, Forest Fire és Erdős-Rényi),
illetve való világból inspirált bemenetekre, ugyanakkor célunk egy új hibrid algoritmus fejlesztése,
melynek eredményei összehasonlíthatóak a standard algoritmusok eredményeivel.

\section{Egycélú CNDP}

\section{Kétcélú CNDP}




{ \renewcommand{\baselinestretch}{0.8}
  \normalsize
  \setlength{\itemsep}{-2.4mm}
  \setlength{\bibspacing}{0.67\baselineskip}
  \bibliographystyle{abbrvnat_hu}
  \bibliography{dolgozat}
}

\end{document}
