% BEÁLLÍTÁSOK - JOBB NEM VÁLTOZTATNI
\documentclass[final]{ubb_dolgozat}
\usepackage{definitions}

% pszeudokód írás végett
\usepackage{algpseudocode}
\usepackage[chapter]{algorithm}

% képek miatt
% \usepackage[justification=centering]{caption}
\usepackage{subcaption}
\captionsetup[table]{skip=10pt}
\captionsetup{subrefformat=parens}

% táblázatok
\usepackage{tabularx}
\usepackage{makecell}
\usepackage{multirow}

% listák
\usepackage[shortlabels]{enumitem}

% hasznos rövidítések
\providecommand{\abs}[1]{\lvert#1\rvert}
\DeclareMathOperator*{\argmin}{arg\,min}
\DeclareMathOperator*{\argmax}{arg\,max}

\providecommand{\keywords}[1]
{
  \small
  \textbf{\textit{Keywords:}} #1
}

% milyen nyelveken akarunk forráskódot megjeleníteni
\lstloadlanguages{Python}
% más lehetőségek:
% C, Matlab, Mathematica, Octave, Pascal, Perl, Python
% SCilab, SQL, Haskell, Lisp, Lua, make, ML, PHP, Prolog
%
% a teljes lista a LISTINGS csomagban.


% ezt be lehet tenni MINDEGYIK megjelenítendő kód elé opcióként
\lstset{language=Python}


%%%%%%%%%%%%%%%%%%%%%%%%%%%%%%%%%%%%%%%%%%%%%%%
%%!!          EZT KELL VÁLTOZTATNI       !!%%%%
%%     A DOLGOZAT CÍMOLDALÁNAK ELEMEI        %%

%% MELYIK ÉVBEN ADJUK LE
\submityear{%
2020
}

\titleHU{%
Kritikus csomópontok meghatározása komplex hálózatokban
}

% Az alábbi sorokat ki kell tölteni!

\titleEN{%
Critical node detection problem in complex networks
}

\titleRO{%
Identificarea nodurilor critice în rețele complexe
}

\author{%
Béczi Eliézer
}

%%
\tutorHU{%
dr. Gaskó Noémi,\newline egyetemi docens\\
% a hozzátartozás akkor szükséges, ha NEM BBTE-s a tanár
%{\large Babe\c{s}--Bolyai Tudományegyetem,\\
% Matematika és Informatika Kar}% ha különbözik, akkor fel kell tűntetni
}
%%
\tutorRO{%
Conf. dr. Gaskó Noémi\\
% az egyetem akkor szükséges, ha nem BBTE-s a tanár, a minta a BBTE-t
% tartalmazza
% {\large Universitatea Babe\c{s}--Bolyai,\\ % dacã diferã!!!
% Facultatea de Matematic\u{a} \c{s}i Informatic\u{a} }%
}
%%
\tutorEN{%
Assoc. prof. dr. Gaskó Noémi
% {\large Babe\c{s}--Bolyai University,\\
% Faculty of Mathematics and Informatics}
}


% 
%\includeonly{bevezet}


\begin{document}

% ez a címoldal része
\maketitle

%% ABSTRACT
\begin{abstractEN} % ANGOL VÁLTOZAT
  {
    \vfill
    The purpose of this thesis is to address the critical node detection problem (CNDP).
    The CNDP is an optimization problem that consists of finding a subset of nodes that if removed from the graph will greatly decrease the connectivity of the network.
    We approach the problem from two different perspectives: from a single-objective and a bi-objective standpoint.
    The single-objective formulation of the CNDP aims to minimize the pairwise connectivity of the graph, whereas the goal of the bi-objective one (BOCNDP) is to maximize the number of connected components while simultaneously minimizing the variance of their cardinalities.
    For solving the CNDP we propose three different algorithms: a greedy, a genetic and a memetic algorithm.
    In the case of the BOCNDP, we use six common MOEAs (NSGA-II, PAES etc.), and experiment with different dominance operators, such as Pareto-, Nash- and Berge-dominance.
    Finally, we give a brief comparison of the algorithms using a benchmark set that contains four groups of graphs with different characteristics (Barabási–Albert, Erdős–Rényi etc.).
    \\[.5cm]
    \keywords{
      complex networks,
      critical node detection problem,
      single-objective,
      bi-objective,
      evolutionary algorithms,
      game theory
    }
    \vfill
  }
  \vspace*{.5cm}
  This work is the result of my own activity. I have neither given nor received unauthorized assistance on this work.
\end{abstractEN}

% a dolgozat tartalomjegyzéke -- ez automatikusan generálódik a STRUKTÚRA alapján.
{
\baselineskip 1ex
\parskip 1ex
\tableofcontents
}


%%%%%%%%%%%%%%%%%%%%%%%%%%%%%%%%%%%%%%%%%%%%%%%%%%%%%%%%%%%
%%%%%%%%%%         a dolgozat tartalma         %%%%%%%%%%%%

% ajánlott külön file-okba írni az egyes fejezeteket,
% ugyanis úgy jobban át lehet látni.


% a bevezető fejezet FILE-ja.
\include{bevezet}

% saját fejezetek
\chapter{Bevezető}\label{ch:BEVEZETO}

\section{Áttekintés}\label{sec:ATTEKINTES}

Hálózatok terén nem minden csomópont egyforma fontosságú.
A kulcsfontosságú csomópontok keresésével hálózatokban széles körben foglalkoznak,
különösképpen olyan csomópontok esetén, melyek a hálózat konnektivitásához köthetők.
Ezeket a csomópontokat általában úgy nevezzük, hogy Kritikus Csomópontok.

Kritikus Csomópontok Meghatározásának Problémája (\textbf{CNDP})
egy optimalizációs feladat, amely egy olyan csoport csomópont
megkereséséből áll, melyek törlése maximálisan rontja a hálózat
konnektivitását bizonyos predefiniált konnektivitási metrikák szerint.

A CNDP számos alkalmazási területtel rendelkezik.
Például, közösségi hálók nagy befolyással bíró egyedeinek azonosítása,
komputációs biológiában kapcsolatok definiálására jelút
vagy fehérje-fehérje kölcsönhatás hálózatokban,
smart grid sebezhetőségének vizsgálata, egyének meghatározása
védőoltással való ellátásra vagy karanténba való zárásra egy
fertőzés terjedésének gátlása érdekében.

A CNDP egy $\mathcal{N}\mathcal{P}$-teljes feladat. Adva van egy $G = (V, E)$ gráf, ahol $|V| = n$ a csomópontok száma,
és $|E| = m$ pedig az élek száma. A feladat $k$ kritikus csomópont meghatározása, amelyek törlése a bemeneti
gráfból minimalizálja a hálózat páronkénti konnektivitását. Az alapján, hogy mit értünk egy hálózat
konnektivitása alatt, a CNDP-nak van egycélú illetve többcélú megfogalmazása is.

\section{Hozzájárulásaink}\label{sec:HOZZAJARULASAINK}

Ebben a dolgozatban a CNDP-t két szemszögből fogjuk megközelíteni: egy egycélú, illetve egy kétcélú nézőpontból.
A CNDP egycélú megfogalmazása esetén a páronkénti konnektivitást szeretnénk minimalizálni, míg a kétcélú CNDP (BOCNDP) esetén maximalizálni akarjuk a gráf összefüggő komponenseinek a számát, de ugyanakkor minimalizálni ezen komponensek számosságainak a varianciáját.
A CNDP megoldására három különböző algoritmust mutatunk be: egy mohó, egy genetikus, valamint egy memetikus algoritmust.
A BOCNDP esetén hat multikritériumú genetikus algoritmust fogunk használni (NSGAII, PAES stb.), amely algoritmusok esetén különböző dominancia operátorokkal kísérletezünk, mint például a Pareto-, Nash- vagy Berge-dominancia.
Továbbá különböző intelligens inicializálási módszereket mutatunk be (DFS, véletlen séta stb.), amelyek a kezdeti populáció egyedeit oly módon építik fel, hogy figyelembe veszik a gráf szerkezetét.
Végezetül az algoritmusokat összehasonlítjuk egymással különböző szintetikus, illetve való világból inspirált bemenetekre (Barabási–Albert, Forest-fire stb.), és a hipertérfogat indikátor segítségével mérjük mindegyik algoritmus teljesítményét.


\chapter{Egycélú CNDP}\label{ch:EGYCELU_CNDP}

\section{Páronkénti konnektivitás}\label{sec:PAIRWISE_CONNECTIVITY}

Egycélú CNDP esetén a kihívás abban áll, hogy találjunk egy olyan konnektivitási metrikát,
amely alkalmazási területtől függően megfelelően leírja egy gráf összefüggőségét.
$S$-el fogjuk jelölni a törlendő csomópontok halmazát,
míg az$f(S)$ jóság függvény fogja jellemezni a $G[V \setminus S]$ feszített részgráf összefüggőségét.
Ha $H$-val jelöljük a $G[V \setminus S]$ feszített részgráf összefüggő komponenseinek a halmazát,
akkor a jóságfüggvény a következő képlettel írható le:
\begin{equation}\label{eqn:PAIRWISE_CONNECTIVITY}
  f(S) = \sum_{h \in H} \frac{|h| \cdot (|h| - 1)}{2},
\end{equation}
amelyet az irodalom \cite{ventresca2012global, aringhieri2016general} úgy tart számon,
hogy \textbf{páronkénti konnektivitás}.
Tehát a feladat \aref{eqn:PAIRWISE_CONNECTIVITY} függvénynek a minimalizálása:
\begin{equation}\label{eqn:MIN_PAIRWISE_CONNECTIVITY}
  \min_{S \subseteq V} f(S).
\end{equation}

\Aref{eqn:PAIRWISE_CONNECTIVITY} fitness függvény implementációját
\aref{lst:PAIRWISE-CONNECTIVITY} kódrészlet szemlélteti Python-ban.
A továbbiakban tárgyalt 3 algoritmus ezt a fitness függvényt fogja használni.
\lstinputlisting[language={Python}, caption={Páronkénti konnektivitás}, label={lst:PAIRWISE-CONNECTIVITY}]{./progfiles/single-objective-cndp/pairwise_connectivity.py}

\section{Mohó algoritmus}\label{sec:MOHO_ALGORITMUS}

\subsection{Általánosan}
Egy mohó algoritmus egy egyszerű és intuitív algoritmus, amely gyakran használt
optimalizációs feladatok megoldására. Az algoritmus helyi optimumok megvalósításával próbálja
megtalálni a globális optimumot.

Habár a mohó algoritmusok jól működnek bizonyos feladatok esetében,
mint pl. Dijkstra-algoritmus, amely egy csomópontból kiindulva meghatározza a legrövidebb utakat,
vagy Huffman-kódolás, amely adattömörítésre szolgál, de sok esetben nem eredményeznek optimális megoldást.
Ez annak köszönhető, hogy míg a mohó algoritmus függhet az előző lépések választásától,
addig a jövőben meghozott döntésektől független.

Az algoritmus minden lépésben mohón választ, folyamatosan lebontva a feladatot kisebb feladattá.
Más szavakkal, a mohó algoritmus soha nem gondolja újra választásait.

\subsection{Saját mohó algoritmus}
A CNDP esetén a mohó algoritmust \aref{lst:GREEDY-ALGORITMUS} kódrészlet szemlélteti.
\lstinputlisting[language={Python}, caption={Saját mohó algoritmus}, label={lst:GREEDY-ALGORITMUS}]{./progfiles/single-objective-cndp/greedy/greedy_algorithm.py}

A mohó algoritmus kiindul a gráf csúcslefedéséből.
\footnote{
  Angolul: vertex cover.
}
Ez lesz a kezdeti $S$ megoldásunk.
A maradék csomópontok $V \setminus S$  a gráf maximális független csúcshalmazát
\footnote{
  Angolul: maximal independent set.
}
\emph{MIS} alkotják.
Mivel majdnem biztos, hogy $|S| > k$, ezért mohón elkezdünk kivenni csomópontokat $S$-ből,
majd ezeket hozzáadni \emph{MIS}-hez, amíg $|S| > k$.
A hozzáadott csomópont az lesz, amelyiket ha visszatesszük az eredeti gráfba,
akkor a minimum értéket téríti vissza a páronkénti konnektivitásra a keletkezett gráfban.

Mivel több olyan csomópont lehet, amelyeket ha visszateszünk az eredeti gráfba,
akkor ugyanazt a minimális értéket adják vissza a páronkénti konnektivitásra,
ezért ezeket eltároljuk a B halmazban, és minden lépésben random módon határozzuk meg,
hogy melyik kerüljön vissza \emph{MIS}-be.

Ezzel az eljárással garantáljuk, hogy a mohó algoritmusunk különböző megoldásokat fog adni
többszöri futtatások esetén.

\section{Genetikus algoritmus}\label{sec:GENETIKUS_ALGORITMUS}

A genetikus algoritmus a metaheurisztikák osztályába tartozik, és a természetes kiválasztódás inspirálta.
Egy globális optimalizáló, amely gyakran használt optimalizációs és keresési problémák esetében,
ahol a sok lehetséges megoldás közül a legjobbat kell megkeresni.
Azt hogy egy megoldás mennyire jó, a fitness függvény mondja meg.

A genetikus algoritmus mindig egy populációnyi megoldással dolgozik.
A populációba egyedek tartoznak, amelyek egyenként egy-egy megoldásai a feladatnak.
Az algoritmus minden iterációban egy új populációt állít elő az aktuális populációból úgy,
hogy a \textbf{szelekciós operátor} által kiválasztott legrátermettebb szülőkön alkalmazza a
\textbf{rekombinációs} és \textbf{mutációs operátorokat}.

Ezen algoritmusok alapötlete az, hogy minden újabb generáció
az előzőnél valamelyest rátermettebb egyedeket tartalmaz, és így a keresés folyamán
egyre jobb megoldások születnek.

\input{chapters/single-objective-cndp/memetic-algorithm}

\chapter{Kétcélú CNDP}\label{ch:KETCELU_CNDP}

\chapter{Következtetések}


Végezetül, egy szép és érdekes probléma a CNDP.
A megoldási módszereket korántsem merítettük ki.
Lehetne kísérletezni más algoritmusokkal, más keresztezési és mutációs operátorok használatával, beépíteni különböző heurisztikákat a genetikus algoritmusba.
A hasonló feladatok megoldására nincs egy általános recept.
Az ember kreativitásán múlik, hogy talál-e egy algoritmust, amely jobb az eddigieknél.
A tudomány ezen ágán, a határ valóban csak a csillagos ég.


{ \renewcommand{\baselinestretch}{0.8}
  \normalsize
  \setlength{\itemsep}{-2.4mm}
  \setlength{\bibspacing}{0.67\baselineskip}
  \bibliographystyle{abbrvnat_hu}
  \bibliography{dolgozat}
}

\end{document}
