%!TEX root = minta_dolgozat.tex
%%%%%%%%%%%%%%%%%%%%%%%%%%%%%%%%%%%%%%%%%%%%%%%%%%%%%%%%%%%%%%%%%%%%%%%
\chapter{Alapok}\label{ch:ALAP}
%%%%%%%%%%%%%%%%%%%%%%%%%%%%%%%%%%%%%%%%%%%%%%%%%%%%%%%%%%%%%%%%%%%%%%%

\begin{osszefoglal}
	A fejezetek elején egy rövid összefoglalót teszünk. Ez a rész opcionális.
	
\end{osszefoglal}

%%%%%%%%%%%%%%%%%%%%%%%%%%%%%%%%%%%%%%%%%%%%%%%%%%%%%%%%%%%%%%%%%%%%%%%
\section{A gépi tanulás}\label{sec:ALAP:ml}

A gépi tanulás neve  \citeN{Mitchell97} azonos címû -- ``{\em Machine Learning}'' -- könyvéből származtatható.
A könyv alapján azt a kutatási területet nevezzük így, amelyben a cél olyan programok írása, amelyek  futtatásuk során fejlődnek, vagyis valamilyen szempont szerint jobbak, okosabbak lesznek.%
\footnote{ %
Részlet \citeN{Mitchell97} könyvéből:\newline
    ``The field of machine learning is concerned with the question of how to construct computer programs    that automatically improve with experience.''
}  %
Itt az ``okosság'' metaforikus: a futási idő folyamán valamilyen mérhető jellemzőnek a javulását értjük alatta.
%
Például a felhasználás kezdetén a szövegfelismerő még nem képes a szövegek azonosítására, azonban a használat -- és a felhasználói utasítások -- után úgy módosítja a működési paramétereit, hogy a karakterek egyre nagyobb hányadát tudja felismerni.


%%%%%%%%%%%%%%%%%%%%%%%%%%%%%%%%%%%%%%%%%%%%%%%%%%%%%%%%%%%%%%%%%%%%%%%
\section{Adatelemzés}\label{sec:ALAP:adatelem}

A mesterséges intelligencia azon módszereit, amelyeket numerikus
vagy {\em
  enyhén strukturált}%
\footnote{%
 {\bf Enyhén} strukturált (nagyon felületesen): az adatok komponensei
  (dimenziók) közötti kapcsolat nem túl bonyolult.
}%
adatokra tudunk alkalmazni, gépi tanulásos módszereknek nevezzük
\cite{Mitchell97}.  A gépi tanulás e meghatározás alapján egy
szerteágazó tudományág, amelynek keretén belül sok módszerről és
ennek megfelelően sok alkalmazási területről beszélhetünk. A
korábban említett neurális modellekkel ellentétben a központban
itt az adatok vannak: azok típusától függően választunk például a
binomiális modell és a normális eloszlás, a fő- vagy
független-komponensek módszere vagy a k-közép és EM
algoritmusok között.
Egyre több adatunk van, azonban az ``információt'' megtalálni egyre nehezebb.%
\footnote{%
    David Donoho, a Stanford Egyetem professzora szerint a XXI. századot az adatok határozzák meg: azok gyûjtése, szállítása, tárolása, megjelenítése, illetve az adatok {\bf felhasználása}.
}%


%%%%%%%%%%%%%%%%%%%%%%%%%%%%%%%%%%%%%%%%%%%%%%%%%%%%%%%%%%%%%%%%%%%%%%%
\section{Szerkesztés}\label{sec:ALAP:szerkeszt}

A következőkben áttekintjük a \LaTeX dokumentumok szerkesztésének alapjait.

Átfogó referenciák a következők:
\begin{description}%
	\item[\cite{LatexNotSoShort}] -- egy \LaTeX gyorstalpaló. A könyvben nagyon
	  célirányosan mutatják be a szerkesztési szabályokat és a fontosabb parancsokat.
	\item[\cite{LatexNotSoShortHU}] -- a gyorstalpaló magyar változata egy BME-s
	  csapat jóvoltából.
	\item[\cite{Doob95,MittelbachEtAl04}] -- az angol nyelvû alapkönyvek. Kérésre az
	  utóbbi -- \cite{MittelbachEtAl04} -- elérhetővé tehető.
\end{description}
Természetesen a Kari könyvtárban is találtok könyvészetet. A fentebb említetteken kívül nagyon sok internetes oldal tartalmaz \LaTeX szerkesztésről bemutatókat:


\begin{figure}[t]
  \centering
  \pgfimage[width=0.2\linewidth]{images/bayes}
  \caption[Példa képek beszúrására]%
  {Példa képek beszúrására: a képen rev. Thomas Bayes látható. A képek után {\em kötelezően} szerepelnie kell a forrásnak:\\
  {\white .}\hfill\url{http://en.wikipedia.org/wiki/Thomas_Bayes}}
  \label{fig:ALAP:sm1}
\end{figure}

Képeket beszúrni a\\
 \verb+\pgfimage[width=0.4\linewidth]{images/bayes}+\\
paranccsal lehet, ahol a\\
\verb+width=0.4\linewidth+ \\
a kép szélességét jelenti. Amennyiben a magasság nincs megadva -- mintjelen esetben -- akkor azt automatikusan számítja ki a rendszer, az eredeti kép arányait figyelembe véve.
Egy másik jellegzetesség az, hogy a képek kiterjesztését nem adjuk meg -- a \LaTeX megkeresi a számára elfogadható kiterjesztéseket, azok listájából az elsőt használja. A .JPG, .PNG, .TIFF, valamint a .PDF kiterjesztések is használhatók.

Két kép egymás mellé tétele a \verb+tabular+ környezet-mintával lehetséges, amint a \ref{fig:ALAP:sm2} ábrán látjuk.
Amennyiben grafikonunk van, általában ajánlott VEKTOROSAN menteni -- ezt általában a PDF driverrel tesszük -- az eredmény a \ref{fig:ALAP:sm3} ábrán látható.

\begin{figure}[t]
  \centering
  \begin{tabular}{ccc}
		  \pgfimage[height=4cm]{images/bayes}
		  &
		  \pgfimage[height=4cm]{images/vapnik}
	\end{tabular}
  \caption[Példa képek beszúrására egy táblában]%
  {Példa képek beszúrására: a bal oldalon rev. Thomas Bayes, a jobb oldalon egy jelenkori matematikus, Vladimir Vapnik látható.\\
  {\white .}\hfill\url{http://en.wikipedia.org/wiki/Vladimir_Vapnik}}
  \label{fig:ALAP:sm2}
\end{figure}

\begin{figure}[t]
  \centering
  \pgfimage[width=0.7\linewidth]{images/parzen}
  \caption[Példa grafika beszúrására]%
  {Példa grafika beszúrására.}
  \label{fig:ALAP:sm3}
\end{figure}

A szerkesztés folyamata során ajánlott a:
\begin{itemize}
	\item különböző strukturális elemek használata: a \verb+\chapter+, \verb+\section+, \verb+\subsection+, \verb+\subsubsection+ parancsok.
	\item a listák használata felsorolásoknál;
  \item a tétel típusú environmentek és a bizonyítás-environment használata (alább bemutatunk néhányat, az összes értelmezett tétel típust megtalálod a definitions.sty fájlban).
\end{itemize}

\begin{ert}
  Az $n$ pozitív egész számot \emph{négyzetmentesnek} nevezzük, ha az $n$ prímtényezős felbontásában minden prím legfentebb az első hatványon szerepel.
\end{ert}
\begin{pld}
  Az 1, 7 és 33 természetes számok négyzetmentesek, míg a 9 és a 45 nem.
\end{pld}
\begin{tet}\label{tet:negyzment}
  Az $n$ pozitív egész szám akkor és csak akkor négyzetmentes, ha minden $n$ elemű Abel csoport ciklikus.
\end{tet}
\begin{proof}
  Túl hosszú!
\end{proof}
\begin{meg}
  \Aref{tet:negyzment} tétel a végesen generált Abel csoportok jellemzési tételének a következménye. 
\end{meg}