\section{Áttekintés}\label{sec:ATTEKINTES}

Hálózatok terén nem minden csomópont egyforma fontosságú.
A kulcsfontosságú csomópontok keresésével hálózatokban széles körben foglalkoznak,
különösképpen olyan csomópontok esetén, melyek a hálózat konnektivitásához köthetők.
Ezeket a csomópontokat általában úgy nevezzük, hogy Kritikus Csomópontok.

Kritikus Csomópontok Meghatározásának Problémája (CNDP)
egy optimalizációs feladat, amely egy olyan csoport csomópont
megkereséséből áll, melyek törlése maximálisan rontja a hálózat
konnektivitását bizonyos predefiniált konnektivitási metrikák szerint.

A CNDP számos alkalmazási területtel rendelkezik.
Például, közösségi hálók nagy befolyással bíró egyedeinek azonosítása,
komputációs biológiában kapcsolatok definiálására jelút
vagy fehérje-fehérje kölcsönhatás hálózatokban,
smart grid sebezhetőségének azonosítása, egyének meghatározása
védőoltással való ellátásra vagy karanténba való zárásra egy
fertőzés terjedésének gátlása érdekében.

A CNDP egy $\mathcal{N}\mathcal{P}$-teljes feladat. Adva van egy $G = (V, E)$ gráf, ahol $|V| = n$ a csomópontok száma,
és $|E| = m$ pedig az élek száma. A feladat $k$ kritikus csomópont meghatározása, amelyek törlése a bemeneti
gráfból minimalizálja a hálózat páronkénti konnektivitását. Az alapján, hogy mit értünk egy hálózat
konnektivitása alatt, a CNDP-nak van egycélú illetve többcélú megfogalmazása is.
