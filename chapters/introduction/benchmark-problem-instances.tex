\section{Bemeneti tesztgráfok}\label{sec:BENCHMARK-PROBLEM-INSTANCES}
Benchmark tesztelés végett a \citeN{ventresca2012global} által javasolt gráfhalmazt fogjuk használni, amelyben négy alapvető típus jelenik meg, mindegyik a maga jellegzetességeivel.
\Aref{tab:BENCHMARK_INSTANCES}. táblázat bemutatja az illető gráfhalmazt, feltüntetve mindegyik bemenet esetén a következő tulajdonságokat:
a példány nevét, a csomópontok $\abs{V}$ és az élek $\abs{E}$ számát, a törlendő (kritikus) csomópontok számát $\left( k \right)$,
a gráf átlagos fokszámát $\langle d \rangle$, átmérőjét ($D$), sűrűségét ($\rho$), modularitását ($Q$) és átlagos távolságát ($l_G$).
A következőkben ezeket a modelleket szeretnénk röviden ismertetni.


\subsection{Barabási–Albert modell}
A Barabási–Albert modell olyan komplex hálózatok struktúráját írja le, amelyek skálafüggetlenek, vagyis a fokszámeloszlás nem változik az idő múlásával.
Két alapgondolatot foglal magában: \textit{folyamatos növekedés} és \textit{preferenciális kapcsolódás}\footnote{ Angolul: preferential attachment. }. Mindkét fogalom széles körben ismert a valós hálózatok terén.
A folyamatos növekedés alatt azt értjük, hogy a hálózat csúcsainak a száma egyre növekszik az idő múlásával, mivel újabb és újabb csúcsokat adunk hozzá a gráfhoz.
A preferenciális kapcsolódás pedig azt jelenti, hogy minél nagyobb a fokszáma egy csomópontnak, annál nagyobb valószínűséggel fog kapcsolódni egy új csomóponthoz.
\Aref{fig:BENCHMARK_INSTANCES}. ábra \subref{fig:BARABASI_ALBERT_MODEL} alábrája bemutat egy $1000$ csomópontból álló Barabási–Albert típusú gráfot.


\subsection{Erdős–Rényi modell}
Az Erdős–Rényi modell véletlen gráfok előállítására szolgáló modell. Két különböző konstrukciót is jelöl:
a $G(n, M)$ modellben $n$ csúcsú és $M$ élű gráfok közül választunk azonos valószínűséggel,
míg a $G(n, p)$ modellben az $n$ csúcsú gráf minden élét, egymástól függetlenül, $p$ valószínűséggel húzzuk be.
\Aref{fig:BENCHMARK_INSTANCES}. ábra \subref{fig:ERDOS_RENYI_MODEL} alábrája bemutat egy $466$ csomópontból és 700 élből álló Erdős–Rényi típusú gráfot.


\subsection{Forest-fire modell}
Mint a Barabási–Albert modell, a Forest-fire is a preferenciális kapcsolódás megközelítést használja,
viszont a csomópontok fokszáma egy \textit{lassan lecsengő eloszlást}\footnote{Angolul: heavy-tailed distribution.} mutat,
a hálózat átmérője csökkenvén az idő múlásával. Ennek eredményeképpen a modell egy \textit{sűrűsödő hatványtörvényt}\footnote{Angolul: densification power-law.} követ,
ami azt jelenti, hogy a hálózat egy hatványtörvénynek megfelelően sűrűsödik.
A modell onnan kapta az elnevezését, hogy növekedésének a mintája egy erdőtűz terjedéséhez hasonlít.
\Aref{fig:BENCHMARK_INSTANCES}. ábra \subref{fig:FOREST_FIRE_MODEL} alábrája bemutat egy $500$ csomópontból álló Forest-fire típusú gráfot.


\subsection{Watts–Strogatz modell}
A Watts–Strogatz modell olyan véletlen gráfok előállítására szolgáló modell, amelyek \textit{kis-világ} tulajdonságúak.
Ezen hálózatok átmérője kicsi, és a legtöbb csomópont elérhető minden más csomópontból relatív kevés ugráson vagy lépésen belül.
Így a csúcsok közötti átlagos távolság rövid.
Továbbá, ezen hálózatok magas klaszterezettségi együtthatóval rendelkeznek, vagyis a csomópontok hajlamosak csoportokba tömörülni.
\Aref{fig:BENCHMARK_INSTANCES}. ábra \subref{fig:WATTS_STROGATZ_MODEL} alábrája bemutat egy $250$ csomópontból álló Watts–Strogatz típusú gráfot.


\begin{table}[b]
  \centering
  \caption{Benchmark tesztelésre használt bemeneti példányok}\label{tab:BENCHMARK_INSTANCES}
  \begin{tabularx}{\textwidth} {
      >{\raggedright\arraybackslash}X
      >{\raggedleft\arraybackslash}X
      >{\raggedleft\arraybackslash}X
      >{\raggedleft\arraybackslash}X
      >{\raggedleft\arraybackslash}X
      >{\raggedleft\arraybackslash}X
      >{\raggedleft\arraybackslash}X
      >{\raggedleft\arraybackslash}X
      >{\raggedleft\arraybackslash}X
    }
    \Xhline{4\arrayrulewidth}
    Gráf   & $\abs{V}$ & $\abs{E}$ & $k$   & $\langle d \rangle$ & $D$  & $\rho$  & $Q$     & $l_G$   \\
    \Xhline{4\arrayrulewidth}
    BA500  & $500$     & $499$     & $50$  & $1.996$             & $13$ & $0.004$ & $0.886$ & $5.663$ \\
    BA1000 & $1000$    & $999$     & $75$  & $1.998$             & $18$ & $0.002$ & $0.910$ & $6.045$ \\
    BA2500 & $2500$    & $2499$    & $100$ & $1.999$             & $17$ & $0.001$ & $0.946$ & $6.901$ \\
    BA5000 & $5000$    & $4999$    & $150$ & $2.000$             & $24$ & $0.000$ & $0.963$ & $8.380$ \\
    \hline
    ER250  & $235$     & $350$     & $50$  & $2.979$             & $14$ & $0.013$ & $0.603$ & $5.338$ \\
    ER500  & $466$     & $700$     & $80$  & $3.004$             & $14$ & $0.006$ & $0.631$ & $5.973$ \\
    ER1000 & $941$     & $1400$    & $140$ & $2.976$             & $16$ & $0.003$ & $0.649$ & $6.558$ \\
    ER2500 & $2344$    & $3500$    & $200$ & $2.986$             & $16$ & $0.001$ & $0.663$ & $7.516$ \\
    \hline
    FF250  & $250$     & $514$     & $50$  & $4.112$             & $14$ & $0.017$ & $0.638$ & $4.816$ \\
    FF500  & $500$     & $828$     & $110$ & $3.312$             & $15$ & $0.007$ & $0.798$ & $6.026$ \\
    FF1000 & $1000$    & $1817$    & $150$ & $3.634$             & $20$ & $0.004$ & $0.793$ & $6.173$ \\
    FF2000 & $2000$    & $3413$    & $200$ & $3.413$             & $19$ & $0.002$ & $0.880$ & $7.587$ \\
    \hline
    WS250  & $250$     & $1246$    & $70$  & $9.968$             & $6$  & $0.040$ & $0.697$ & $3.327$ \\
    WS500  & $500$     & $1496$    & $125$ & $5.984$             & $10$ & $0.012$ & $0.789$ & $5.304$ \\
    WS1000 & $1000$    & $4996$    & $200$ & $9.992$             & $7$  & $0.010$ & $0.803$ & $4.444$ \\
    WS2000 & $1500$    & $4498$    & $265$ & $5.997$             & $15$ & $0.004$ & $0.872$ & $7.554$ \\
    \Xhline{4\arrayrulewidth}
  \end{tabularx}
\end{table}


\begin{figure}[ht]
  \begin{subfigure}[b]{0.5\linewidth}
    \centering
    \includegraphics[width=1\linewidth]{images/barabasi_albert_model.png}
    \caption{ Barabási–Albert típusú gráf, $1000$ csomópont. }
    \label{fig:BARABASI_ALBERT_MODEL}
    \vspace{4ex}
  \end{subfigure}%%
  \begin{subfigure}[b]{0.5\linewidth}
    \centering
    \includegraphics[width=1\linewidth]{images/erdos_renyi_model.png}
    \caption{ Erdős–Rényi típusú gráf, $466$ csomópont. }
    \label{fig:ERDOS_RENYI_MODEL}
    \vspace{4ex}
  \end{subfigure}
  \begin{subfigure}[b]{0.5\linewidth}
    \centering
    \includegraphics[width=1.15\linewidth]{images/forest_fire_model.png}
    \caption{ Forest-fire típusú gráf, $500$ csomópont. }
    \label{fig:FOREST_FIRE_MODEL}
  \end{subfigure}%%
  \begin{subfigure}[b]{0.5\linewidth}
    \centering
    \includegraphics[width=0.85\linewidth]{images/watts_strogatz_model.png}
    \caption{ Watts–Strogatz típusú gráf, $250$ csomópont. }
    \label{fig:WATTS_STROGATZ_MODEL}
  \end{subfigure}\\
  \caption{A bemeneti példányok négy különböző modellje.}
  \label{fig:BENCHMARK_INSTANCES}
\end{figure}
