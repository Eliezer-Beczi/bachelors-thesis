\section{Hozzájárulásaink}\label{sec:HOZZAJARULASAINK}

Ebben a dolgozatban a CNDP-t két szemszögből fogjuk megközelíteni: egy egycélú, illetve egy kétcélú nézőpontból.
A CNDP egycélú megfogalmazása esetén a páronkénti konnektivitást szeretnénk minimalizálni, míg a kétcélú CNDP (BOCNDP) esetén maximalizálni akarjuk a gráf összefüggő komponenseinek a számát, de ugyanakkor minimalizálni ezen komponensek számosságainak a varianciáját.
A CNDP megoldására három különböző algoritmust mutatunk be: egy mohó, egy genetikus, valamint egy memetikus algoritmust.
A BOCNDP esetén hat multikritériumú genetikus algoritmust fogunk használni (NSGAII, PAES stb.), amely algoritmusok esetén különböző dominancia operátorokkal kísérletezünk, mint például a Pareto-, Nash- vagy Berge-dominancia.
Továbbá különböző intelligens inicializálási módszereket mutatunk be (DFS, véletlen séta stb.), amelyek a kezdeti populáció egyedeit oly módon építik fel, hogy figyelembe veszik a gráf szerkezetét.
Végezetül az algoritmusokat összehasonlítjuk egymással különböző szintetikus, illetve való világból inspirált bemenetekre (Barabási–Albert, Forest-fire stb.), és a hipertérfogat indikátor segítségével mérjük mindegyik algoritmus teljesítményét.
