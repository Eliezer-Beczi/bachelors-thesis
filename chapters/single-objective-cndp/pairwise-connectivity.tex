\section{Páronkénti konnektivitás}\label{sec:PAIRWISE_CONNECTIVITY}

Egycélú CNDP esetén a kihívás abban áll, hogy találjunk egy olyan konnektivitási metrikát,
amely alkalmazási területtől függően megfelelően leírja egy gráf összefüggőségét.
$S$-el fogjuk jelölni a törlendő csomópontok halmazát,
míg az$f(S)$ jóság függvény fogja jellemezni a $G[V \setminus S]$ feszített részgráf összefüggőségét.
Ha $H$-val jelöljük a $G[V \setminus S]$ feszített részgráf összefüggő komponenseinek a halmazát,
akkor a jóságfüggvény a következő képlettel írható le:
\begin{equation}\label{eqn:PAIRWISE_CONNECTIVITY}
  f(S) = \sum_{h \in H} \frac{|h| \cdot (|h| - 1)}{2},
\end{equation}
amelyet az irodalom \cite{ventresca2012global, aringhieri2016general} úgy tart számon,
hogy \textbf{páronkénti konnektivitás}.
Tehát a feladat \aref{eqn:PAIRWISE_CONNECTIVITY} függvénynek a minimalizálása:
\begin{equation}\label{eqn:MIN_PAIRWISE_CONNECTIVITY}
  \min_{S \subseteq V} f(S).
\end{equation}

\Aref{eqn:PAIRWISE_CONNECTIVITY} fitnesz függvény implementációját
\aref{lst:PAIRWISE-CONNECTIVITY} kódrészlet szemlélteti Python-ban.
A továbbiakban tárgyalt 3 algoritmus ezt a fitnesz függvényt fogja használni.
\lstinputlisting[language={Python}, caption={Páronkénti konnektivitás}, label={lst:PAIRWISE-CONNECTIVITY}]{./progfiles/single-objective-cndp/pairwise_connectivity.py}
