\section{Genetikus algoritmus}\label{sec:GENETIKUS_ALGORITMUS}


\subsection{Általánosan}
A genetikus algoritmus a metaheurisztikák osztályába tartozik, és a természetes kiválasztódás inspirálta.
Egy globális optimalizáló, amely gyakran használt optimalizációs és keresési problémák esetében,
ahol a sok lehetséges megoldás közül a legjobbat kell megkeresni.
Azt hogy egy megoldás mennyire jó, a fitness függvény mondja meg.

A genetikus algoritmus mindig egy populációnyi megoldással dolgozik.
A populációba egyedek tartoznak, amelyek egyenként egy-egy megoldásai a feladatnak.
Az algoritmus minden iterációban egy új populációt állít elő az aktuális populációból úgy,
hogy a \textbf{szelekciós operátor} által kiválasztott legrátermettebb szülőkön alkalmazza a
\textbf{rekombinációs} és \textbf{mutációs operátorokat}.

Ezen algoritmusok alapötlete az, hogy minden újabb generáció
az előzőnél valamelyest rátermettebb egyedeket tartalmaz, és így a keresés folyamán
egyre jobb megoldások születnek.

\subsection{Saját genetikus algoritmus}
A CNDP esetén a genetikus algoritmust \aref{lst:GENETIKUS-ALGORITMUS} kódrészlet szemlélteti.
\lstinputlisting[language={Python}, caption={Saját genetikus algoritmus}, label={lst:GENETIKUS-ALGORITMUS}]{./progfiles/single-objective-cndp/ga/genetic_algorithm.py}

Egy Genetikus Algoritmus (GA) standard algoritmikus keretrendszerét használjuk fel.
Generálunk egy kezdeti populációt megoldásokkal. Utána keresztezzük őket, hogy új megoldásokat kapjunk,
amelyeket pedig mutálunk. Ezután rendezzük a régi és új megoldásokat egy fitness függvény alapján,
és létrehozunk egy új populációt eltávolítva a rossz megoldásokat.
A folyamatot addig ismételjük, amíg az iterációk száma el nem ér egy felső korlátot.
Az algoritmus végén visszatérítjük a legjobb megoldást.

\subsubsection{Inicializáció}
A kezdeti populáció egyedeit random generáljuk ki. Ez azt jelenti,
hogy minden egyed kromoszómája egy $k$ csomópontból álló részhalmaza lesz
a bemeneti gráf csomóponthalmazának. Ezt szemlélteti a \aref{lst:GENETIKUS-ALGORITMUS:RANDOM-SOLUTION} kódrészlet.
\lstinputlisting[language={Python}, caption={Random inicializáció}, label={lst:GENETIKUS-ALGORITMUS:RANDOM-SOLUTION}]{./progfiles/single-objective-cndp/ga/random_solution.py}

Egy új fitness függvényt vezetünk be egy-egy egyed jóságának felmérése végett.
Ez abban tér el \aref{sec:PAIRWISE_CONNECTIVITY} részben tárgyaltaktól,
hogy nem csak a páronkénti konnektivitás mértékét vesszük figyelembe egy egyed esetén,
hanem hogy az eddigi talált legjobb megoldástól mennyire tér el.
Ezt a fitness függvényt a következő képlettel írjuk le:
\begin{equation}\label{eqn:SOCNDP_GA_FITNESS_FUNCTION}
  g(S, S^{*}) = f(S) + \gamma \cdot |S \cap S^{*}|.
\end{equation}
A képletben szereplő $S^{*}$ jelenti az eddig talált legjobb megoldást.
A $\gamma$ egy változó, amely abban segít, hogy fenntartsuk a változatosságot a populáció egyedei között,
megbüntetve azokat, amelyek túl közel vannak a legjobbhoz.
A $\gamma$ változót minden iterációban a következő képlettel számoljuk újra:
\begin{equation}\label{eqn:SOCNDP_GA_GAMMA}
  \gamma = \frac{\alpha \cdot f(S)}{\langle |S \cap S^{*}| \rangle_{S \in P}},
\end{equation}
ahol a nevező a populáció egyedeinek átlagos hasonlóságát jelenti a legjobb egyedhez.

\Aref{eqn:SOCNDP_GA_GAMMA} képlet implementációját láthatjuk a \aref{lst:GENETIKUS-ALGORITMUS:UPDATE} kódrészletben.
\lstinputlisting[language={Python}, caption={$\gamma$ inicializálása}, label={lst:GENETIKUS-ALGORITMUS:UPDATE}]{./progfiles/single-objective-cndp/ga/update.py}

\subsubsection{Reprodukció}
