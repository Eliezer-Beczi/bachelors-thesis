\section{Genetikus algoritmus}\label{sec:GENETIKUS_ALGORITMUS}


\subsection{Általánosan}
A genetikus algoritmus a metaheurisztikák osztályába tartozik, és a természetes kiválasztódás inspirálta.
Egy globális optimalizáló, amely gyakran használt optimalizációs és keresési problémák esetében,
ahol a sok lehetséges megoldás közül a legjobbat kell megkeresni.
Azt hogy egy megoldás mennyire jó, a fitnesz függvény mondja meg.

A genetikus algoritmus mindig egy populációnyi megoldással dolgozik.
A populációba egyedek tartoznak, amelyek egyenként  egyed esetén megoldásai a feladatnak.
Az algoritmus minden iterációban egy új populációt állít elő az aktuális populációból úgy,
hogy a \textbf{szelekciós operátor} által kiválasztott legrátermettebb szülőkön alkalmazza a
\textbf{rekombinációs} és \textbf{mutációs operátorokat}.

Ezen algoritmusok alapötlete az, hogy minden újabb generáció
az előzőnél valamelyest rátermettebb egyedeket tartalmaz, és így a keresés folyamán
egyre jobb megoldások születnek.


\subsection{Saját genetikus algoritmus}
A CNDP esetén a genetikus algoritmust \aref{lst:GENETIKUS-ALGORITMUS} kódrészlet szemlélteti.
\lstinputlisting[language={Python}, caption={Saját genetikus algoritmus}, label={lst:GENETIKUS-ALGORITMUS}]{./progfiles/single-objective-cndp/ga/genetic_algorithm.py}

Egy Genetikus Algoritmus (GA) standard algoritmikus keretrendszerét használjuk fel.
Generálunk egy kezdeti populációt megoldásokkal. Utána keresztezzük őket, hogy új megoldásokat kapjunk,
amelyeket pedig mutálunk. Ezután rendezzük a régi és új megoldásokat egy fitnesz függvény alapján,
és létrehozunk egy új populációt eltávolítva a rossz megoldásokat.
A folyamatot addig ismételjük, amíg az iterációk száma el nem ér egy felső korlátot.
Az algoritmus végén visszatérítjük a legjobb megoldást.


\subsubsection{Inicializáció}
A kezdeti populáció egyedeit random generáljuk ki. Ez azt jelenti,
hogy minden egyed kromoszómája egy $k$ csomópontból álló részhalmaza lesz
a bemeneti gráf csomóponthalmazának. Ezt szemlélteti \aref{lst:GENETIKUS-ALGORITMUS:RANDOM-SOLUTION} kódrészlet.
\lstinputlisting[language={Python}, caption={Random inicializáció}, label={lst:GENETIKUS-ALGORITMUS:RANDOM-SOLUTION}]{./progfiles/single-objective-cndp/ga/random_solution.py}

Egy új fitnesz függvényt vezetünk be  egyed esetén egyed jóságának felmérése végett.
Ez abban tér el \aref{sec:PAIRWISE_CONNECTIVITY} részben tárgyaltaktól,
hogy nem csak a páronkénti konnektivitás mértékét vesszük figyelembe egy egyed esetén,
hanem hogy az eddigi talált legjobb megoldástól mennyire tér el.
Ezt a fitnesz függvényt a következő képlettel írjuk le:
\begin{equation}\label{eqn:SOCNDP_GA_FITNESZ_FUNCTION}
  g(S, S^{*}) = f(S) + \gamma \cdot |S \cap S^{*}|.
\end{equation}
A képletben szereplő $S^{*}$ jelenti az eddig talált legjobb megoldást.
A $\gamma$ egy változó, amely abban segít, hogy fenntartsuk a változatosságot a populáció egyedei között,
megbüntetve azokat, amelyek túl közel vannak a legjobbhoz.
A $\gamma$ változót minden iterációban a következő képlettel számoljuk újra:
\begin{equation}\label{eqn:SOCNDP_GA_GAMMA}
  \gamma = \frac{\alpha \cdot f(S)}{\langle |S \cap S^{*}| \rangle_{S \in P}},
\end{equation}
ahol a nevező a populáció egyedeinek és a legjobb egyed közötti átlagos hasonlóságot fejezi ki.
Az $\alpha$ pedig a képletben található változók egymás feletti fontosságát befolyásolja.
\Aref{eqn:SOCNDP_GA_GAMMA} képlet implementációját \aref{lst:GENETIKUS-ALGORITMUS:UPDATE} kódrészlet mutatja be.
\lstinputlisting[language={Python}, caption={$\gamma$ inicializálása}, label={lst:GENETIKUS-ALGORITMUS:UPDATE}]{./progfiles/single-objective-cndp/ga/update.py}

A $\pi$ paraméter a mutáció valószínűségét fejezi ki egy egyed esetén.
Ezt kezdetben $\pi_{min}$-re állítjuk, de minden iterációban frissítjük aszerint,
hogy találtunk-e az új generációban egy olyan megoldást, amely jobb, mint a globális legjobb.
Ha találtunk az eddigieknél jobb megoldást, akkor a $\pi$ értékét $\pi_{min}$-re állítjuk,
különben a $\pi = \min \left(\pi + \Delta \pi, \pi_{max} \right)$ képlet szerint növeljük.
Ez arra jó, hogy fenntartsuk a populáció sokféleségét abban az esetben,
amikor nem tudunk javítani az eddig talált legjobb megoldáson,
mindezt úgy, hogy megnöveljük a mutációk kialakulásának a valószínűségét.


\subsubsection{Reprodukció}
A genetikus algoritmus egy kulcsfontosságú fázisa a reprodukció.
Itt döntjük el, hogy a meglévő populációból miként jöjjön létre az új generáció.
Ez azt jelenti, hogy meghatározzuk, hogy az $S_{1}$ és $S_{2}$ szülők kromoszómáit
hogyan olvasztjuk egybe annak érdekében, hogy egy új $S'$ egyed szülessen.
Ezt a folyamatot szemlélteti \aref{lst:GENETIKUS-ALGORITMUS:REPRODUKCIO} kódrészlet.
\lstinputlisting[language={Python}, caption={Reprodukció}, label={lst:GENETIKUS-ALGORITMUS:REPRODUKCIO}]{./progfiles/single-objective-cndp/ga/new_generation.py}

Esetünkben úgy történik egy új egyed létrehozása, hogy random módon kiválasztunk 2 különböző szülőt,
és ezek kromoszómáit egybevonjuk: $S' = S_{1} \cup S_{2}$.
Mivel majdnem biztos, hogy az így kapott egyed kromoszómája több, mint $k$ csomópontot tartalmaz,
ezért szükséges törölnünk belőle nódusokat, amíg $|S'| > k$.
Az hogy melyik nódus kerül törlésre az új egyed kromoszómájából, random módon történik.

Fontos megemlítenünk, hogy mivel a szülőket random módon választjuk ki  egyed esetén egyed létrehozásához,
ezért a populáció egyedei között nem teszünk különbséget.
Vagyis keresztezéskor nem nézzük, hogy csak a legrátermettebb szülőket válasszuk,
hanem egyenlő eséllyel választunk kevésbé jó fitnesz értékkel rendelkező egyedet is szülőnek.
Ez lelassítja a populáció uniformizálódásának folyamatát, de segíti a megoldástér bejárását.
Ez azért jó, mert nem tudjuk előre, hogy a csomópontok mely kombinációja
fogja eredményezni a bemeneti gráf maximális szétesését, ha ezeket együtt töröljük a gráfból.
Ezért a kevésbé jó fitnesz értékkel rendelkező egyedeket sem kell figyelmen kívül hagyni,
mert kombinálva őket jó megoldásokhoz juthatunk.


\subsubsection{Mutáció}
A következő nagy jelentőséggel bíró fázisa a genetikus algoritmusnak a mutáció.
Mutáció alatt azt értjük, hogy vesszük az újonnan létrejött populációt,
és a populációban található egyedek génjeit perturbáljuk valamilyen csekély valószínűséggel.
A mutáció azért tartozik a nagy döntések halmazába,
mert a mutáció révén fenntartjuk a populáció sokféleségét, és elkerüljük a korai konvergenciát.
\footnote{
  Angolul: premature convergence.
}
\Aref{lst:GENETIKUS-ALGORITMUS:MUTACIO} kódrészlet a mutáció műveletét hívatott bemutatni.
\lstinputlisting[language={Python}, caption={Mutáció}, label={lst:GENETIKUS-ALGORITMUS:MUTACIO}]{./progfiles/single-objective-cndp/ga/mutation.py}

A populáció minden egyes új egyede esetén, a mutáció valószínűségét a $\pi$ paraméter befolyásolja.
Generálunk egy egyenletes eloszlású véletlen számot $1$ és $100$ között, és ha ez kisebb, mint $\pi$, akkor módosítjuk a megoldást.
A módosítás úgy történik, hogy leszögezzük, hogy a megoldás hány génjét szeretnénk változtatni.
Ezt a számot tükrözi az $n_{g}$ változó, amely értékét a $\left[0, k\right]$ intervallumból veszi, és random generáljuk.
A következő lépés, hogy kitörlünk $n_{g}$ csomópontot a megoldásból, de mivel majdnem biztos,
hogy a megoldásunk így nem-optimális, mert $|S| < k$, ezért szükséges visszaadogatnunk csomópontokat $S$-be.
Ennek érdekében véletlenszerűen kiválasztunk egy csomópontot a $V \setminus S$ halmazból,
és a kiválasztott csomópontot visszatesszük a megoldásba.


\subsubsection{Szelekció}
Az utolsó fázisa a genetikus algoritmusunknak a szelekció.
Itt döntjük el, hogy mely egyedek fogják alkotni a következő nemzedéket.
Jelen esetben ez úgy megy végbe, hogy összefésüljük a régi $P$ és az újonnan létrejött $P'$ populációkat,
és rendezzük az egyedeket \aref{eqn:SOCNDP_GA_FITNESZ_FUNCTION} fitnesz függvény alapján.
Növekvő sorrendbe rendezzük őket, mivel nem szabad elfelejtenünk, hogy célunk végső soron a páronkénti konnektivitás minimalizálása.
Ezután kiválasztjuk az első $\mathcal{N}$ egyedet, és ezeket visszük tovább a következő iterációba.
