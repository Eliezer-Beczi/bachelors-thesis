\section{Genetikus algoritmus}\label{sec:GENETIKUS_ALGORITMUS}


\subsection{Általánosan}
A genetikus algoritmus a metaheurisztikák osztályába tartozik, és a természetes kiválasztódás inspirálta.
Egy globális optimalizáló, amely gyakran használt optimalizációs és keresési problémák esetében,
ahol a sok lehetséges megoldás közül a legjobbat kell megkeresni.
Azt hogy egy megoldás mennyire jó, a fitnesz vagy jóság függvény mondja meg.

A genetikus algoritmus mindig egy populációnyi megoldással dolgozik.
A populációba egyedek tartoznak, amelyek egyenként megoldásai a feladatnak.
Az algoritmus minden iterációban egy új populációt állít elő az aktuális populációból úgy,
hogy a \textbf{szelekciós operátor} által kiválasztott legrátermettebb szülőkön alkalmazza a
\textbf{rekombinációs} és \textbf{mutációs operátorokat}.

Ezen algoritmusok alapötlete az, hogy minden újabb generáció
az előzőnél valamelyest rátermettebb egyedeket tartalmaz, és így a keresés folyamán
egyre jobb megoldások születnek.


\subsection{Saját genetikus algoritmus}
Egy Genetikus Algoritmus (GA) standard algoritmikus keretrendszerét használjuk fel.
Generálunk egy kezdeti populációt megoldásokkal. Utána keresztezzük őket, hogy új megoldásokat kapjunk,
amelyeket pedig mutálunk. Ezután rendezzük a régi és új megoldásokat egy fitnesz függvény alapján,
és létrehozunk egy új populációt eltávolítva a rossz megoldásokat.
A folyamatot addig ismételjük, amíg az iterációk száma el nem ér egy felső korlátot.
Az algoritmus végén visszatérítjük a legjobb megoldást.
A CNDP esetén a genetikus algoritmust \aref{alg:GENETIKUS_ALGORITMUS} kódrészlet szemlélteti.
\begin{algorithm}[t]
  \caption{Genetic Algorithm}\label{alg:GENETIKUS_ALGORITMUS}
  \begin{algorithmic}[1]
    \Function{GA}{$G, k, N, \pi_{\min}, \pi_{\max}, \Delta \pi, \alpha, t_{\max}$}
    \State $t \leftarrow 0$
    \State \Call{Init}{$N, P, S^{*}, \gamma, \pi$}

    \While{$t < t_{t_{\max}}$}
    \State $P' \leftarrow \Call{Crossover}{k, N, P}$
    \State $P' \leftarrow \Call{Mutation}{k, N, P', \pi}$
    \State $P \leftarrow \Call{Selection}{N, P, P'}$
    \State $S^{*}, \gamma, \pi = \Call{Update}{N, P, S^{*}, \pi, \pi_{\min}, \pi_{\max}, \Delta \pi, \alpha}$
    \State $t \leftarrow t + 1$
    \EndWhile

    \State \Return $P$
    \EndFunction
  \end{algorithmic}
\end{algorithm}



\subsubsection{Reprezentáció}
Egy egyedet egy halmaznyi paraméter (változó) jellemez, amelyeket úgy nevezünk, hogy \textit{gének}.
A gének összessége alkotja a \textit{kromoszómát}, amely a probléma egy lehetséges megoldását kódolja, amit a genetikus algoritmus próbál megoldani.
A CNDP esetén minden gén a bemeneti gráf egy különböző csomópontja lesz, a kromoszóma pedig a gráf csomóponthalmazának egy részhalmaza.
Például, ha a $G$ bemeneti gráf a $V = \left\{ 1, 2, \dotsc, 9 \right\}$ csomópontokból áll, és $k = 5$ nódust akarunk törölni,
akkor a kromoszómát egy öt elemű halmazzal fogjuk reprezentálni: $\left\{g_1,g_2,g_3,g_4,g_5\right\}$.


\subsubsection{Inicializáció}
A kezdeti populáció egyedeit random generáljuk ki.
Ez azt jelenti, hogy minden egyed kromoszómája egy $k$ csomópontból álló részhalmaza lesz a bemeneti gráf csomóponthalmazának.
Ezt szemlélteti \aref{alg:GENETIKUS-ALGORITMUS:RANDOM-SOLUTION} kódrészlet.
\begin{algorithm}
  \caption{Random Solution}\label{alg:GENETIKUS-ALGORITMUS:RANDOM-SOLUTION}
  \begin{algorithmic}[1]
    \Function{Rand Sol}{$k$}
    \State $S \leftarrow V$

    \While{$\abs{S} > k$}
    \State $elem \leftarrow \Call{Select}{S}$
    \State $S \leftarrow S \setminus \left\{ elem \right\}$
    \EndWhile

    \State \Return $S$
    \EndFunction
  \end{algorithmic}
\end{algorithm}


Egy új fitnesz függvényt vezetünk be egy egyed jóságának felmérése végett.
Ez abban tér el \aref{sec:PAIRWISE_CONNECTIVITY} részben tárgyaltaktól,
hogy nem csak a páronkénti konnektivitás mértékét vesszük figyelembe egy egyed esetén,
hanem hogy az eddigi talált legjobb megoldástól mennyire tér el.
Ezt a fitnesz függvényt a következő képlettel írjuk le:
\begin{equation}\label{eqn:SOCNDP_GA_FITNESZ_FUNCTION}
  g(S, S^{*}) = f(S) + \gamma \cdot \abs{S \cap S^{*}}.
\end{equation}
A képletben szereplő $S^{*}$ jelenti az eddig talált legjobb megoldást.
A $\gamma$ egy változó, amely abban segít, hogy fenntartsuk a változatosságot a populáció egyedei között,
megbüntetve azokat, amelyek túl közel vannak a legjobbhoz.
A $\gamma$ változót minden iterációban a következő képlettel számoljuk újra:
\begin{equation}\label{eqn:SOCNDP_GA_GAMMA}
  \gamma = \frac{\alpha \cdot f(S^{*})}{\langle \abs{S \cap S^{*}} \rangle_{S \in P}},
\end{equation}
ahol a nevező a populáció egyedeinek és a legjobb egyed közötti átlagos hasonlóságot fejezi ki.
Az $\alpha$ pedig a képletben található változók egymás feletti fontosságát befolyásolja.

A $\pi$ paraméter a mutáció valószínűségét fejezi ki egy egyed esetén.
Ezt kezdetben $\pi_{\min}$-re állítjuk, de minden iterációban frissítjük aszerint,
hogy találtunk-e az új generációban egy olyan megoldást, amely jobb, mint a globális legjobb.
Ha találtunk az eddigieknél jobb megoldást, akkor a $\pi$ értékét $\pi_{\min}$-re állítjuk,
különben a $\pi = \min \left(\pi + \Delta \pi, \pi_{\max} \right)$ képlet szerint növeljük.
Ez arra jó, hogy fenntartsuk a populáció sokféleségét abban az esetben,
amikor nem tudunk javítani az eddig talált legjobb megoldáson,
mindezt úgy, hogy megnöveljük a mutációk kialakulásának a valószínűségét.

Az $S^{*}$, $\gamma$ és $\pi$ változók frissítését \aref{alg:GENETIKUS-ALGORITMUS:UPDATE} kódrészlet mutatja be.
\begin{algorithm}
  \caption{Update $S^{*}$, $\gamma$ and $\pi$ variables}\label{alg:GENETIKUS-ALGORITMUS:UPDATE}
  \begin{algorithmic}[1]
    \Function{Update}{$N, P, S^{*}, \pi, \pi_{\min}, \pi_{\max}, \Delta \pi, \alpha$}
    \State $avg \leftarrow 0$

    \For{$i \leftarrow 1, N$}
    \State $S \leftarrow P\left[ i \right]$
    \State $avg \leftarrow avg + \abs{S \cap S^{*}}$
    \EndFor

    \State $avg \leftarrow \dfrac{avg}{N}$
    \State $\gamma \leftarrow \dfrac{\alpha \cdot f(S^{*})}{avg}$
    \State $S \leftarrow P\left[ 0 \right]$

    \If{$f(S) < f(S^{*})$}
    \State $S^{*} \leftarrow S$
    \State $\pi \leftarrow \pi_{\min}$
    \Else
    \State $\pi \leftarrow \min(\pi + \Delta \pi, \pi_{\max})$
    \EndIf

    \State \Return $S^{*}, \gamma, \pi$
    \EndFunction
  \end{algorithmic}
\end{algorithm}



\subsubsection{Reprodukció}
A genetikus algoritmus egy kulcsfontosságú fázisa a reprodukció.
Itt döntjük el, hogy a meglévő populációból miként jöjjön létre az új generáció.
Ez azt jelenti, hogy meghatározzuk, hogy az $S_{1}$ és $S_{2}$ szülők kromoszómáit
hogyan olvasztjuk egybe annak érdekében, hogy egy új $S'$ egyed szülessen.

Esetünkben úgy történik egy új egyed létrehozása, hogy random módon kiválasztunk 2 különböző szülőt,
és ezek kromoszómáit egybevonjuk: $S' = S_{1} \cup S_{2}$.
Mivel majdnem biztos, hogy az így kapott egyed kromoszómája több, mint $k$ csomópontot tartalmaz,
ezért szükséges törölnünk belőle nódusokat, amíg $\abs{S'} > k$.
Az hogy melyik nódus kerül törlésre az új egyed kromoszómájából, random módon történik.
A reprodukciós folyamatot \aref{alg:GENETIKUS-ALGORITMUS:REPRODUKCIO} kódrészlet szemlélteti.
\begin{algorithm}[h]
  \caption{Recombination Operator}\label{alg:GENETIKUS-ALGORITMUS:REPRODUKCIO}
  \begin{algorithmic}[1]
    \Function{Crossover}{$k, N, P$}
    \State $P' \leftarrow \emptyset$

    \For{$i \leftarrow 1, N$}
    \State $S_{1} \leftarrow \Call{Select}{P}$
    \State $S_{2} \leftarrow \Call{Select}{P}$
    \State $S' \leftarrow S_{1} \cup S_{2}$

    \If{$\abs{S'} = k$}
    \State $P' \leftarrow P' \cup \left\{ S' \right\}$
    \Else
    \State $S' \leftarrow \Call{Random Sample}{S', k}$\Comment{Take $k$ random elements from $S'$}
    \State $P' \leftarrow P' \cup \left\{ S' \right\}$
    \EndIf
    \EndFor

    \State \Return $P'$
    \EndFunction
  \end{algorithmic}
\end{algorithm}


Fontos megemlítenünk, hogy mivel a szülőket random módon választjuk ki  egyed esetén egyed létrehozásához,
ezért a populáció egyedei között nem teszünk különbséget.
Vagyis keresztezéskor nem nézzük, hogy csak a legrátermettebb szülőket válasszuk,
hanem egyenlő eséllyel választunk kevésbé jó fitnesz értékkel rendelkező egyedet is szülőnek.
Ez lelassítja a populáció uniformizálódásának folyamatát, de segíti a megoldástér bejárását.
Ez azért jó, mert nem tudjuk előre, hogy a csomópontok mely kombinációja
fogja eredményezni a bemeneti gráf maximális szétesését, ha ezeket együtt töröljük a gráfból.
Ezért a kevésbé jó fitnesz értékkel rendelkező egyedeket sem kell figyelmen kívül hagyni,
mert kombinálva őket jó megoldásokhoz juthatunk.


\subsubsection{Mutáció}
A következő nagy jelentőséggel bíró fázisa a genetikus algoritmusnak a mutáció.
Mutáció alatt azt értjük, hogy vesszük az újonnan létrejött populációt,
és a populációban található egyedek génjeit perturbáljuk valamilyen csekély valószínűséggel.
A mutáció azért tartozik a nagy döntések halmazába,
mert a mutáció révén fenntartjuk a populáció sokféleségét, és elkerüljük a korai konvergenciát.
\footnote{
  Angolul: premature convergence.
}

A populáció minden egyes új egyede esetén, a mutáció valószínűségét a $\pi$ paraméter befolyásolja.
Generálunk egy egyenletes eloszlású véletlen számot $1$ és $100$ között, és ha ez kisebb, mint $\pi$, akkor módosítjuk a megoldást.
A módosítás úgy történik, hogy leszögezzük, hogy a megoldás hány génjét szeretnénk változtatni.
Ezt a számot tükrözi az $n_{g}$ változó, amely értékét a $\left[0, k\right]$ intervallumból veszi, és random generáljuk.
A következő lépés, hogy kitörlünk $n_{g}$ csomópontot a megoldásból, de mivel majdnem biztos,
hogy a megoldásunk így nem-optimális, mert $\abs{S} < k$, ezért szükséges visszaadogatnunk csomópontokat $S$-be.
Ennek érdekében véletlenszerűen kiválasztunk egy csomópontot a $V \setminus S$ halmazból,
és a kiválasztott csomópontot visszatesszük a megoldásba.
\Aref{alg:GENETIKUS_ALGORITMUS:MUTACIO} kódrészlet a mutáció műveletét hívatott bemutatni.
\begin{algorithm}
  \caption{Mutation Operator}\label{alg:GENETIKUS_ALGORITMUS:MUTACIO}
  \begin{algorithmic}[1]
    \Procedure{Mutation}{$k, N, P', \pi$}
    \For{$i \leftarrow 1, N$}
    \State $r \leftarrow \Call{Rand Int}{1, 100}$

    \If{$r \leq \pi$}
    \State $S' \leftarrow P'[i]$
    \State $n_{g} \leftarrow \Call{Rand Int}{0, k}$\Comment{Number of genes to mutate}

    \For{$j \leftarrow 1, n_{g}$}
    \State $idx \leftarrow \Call{Rand Int}{1, \abs{S'}}$
    \State $S' \leftarrow S' \setminus \left\{ P\left[ idx \right] \right\}$
    \EndFor

    \State $\textit{MIS} \leftarrow V \setminus S'$

    \While{$\abs{S'} < k$}
    \State $elem \leftarrow \Call{Select}{\textit{MIS}}$
    \State $S' \leftarrow S' \cup \left\{ elem \right\}$
    \EndWhile
    \EndIf
    \EndFor
    \EndProcedure
  \end{algorithmic}
\end{algorithm}



\subsubsection{Szelekció}
Az utolsó fázisa a genetikus algoritmusunknak a szelekció.
Itt döntjük el, hogy mely egyedek fogják alkotni a következő nemzedéket.
Jelen esetben ez úgy megy végbe, hogy összefésüljük a régi $P$ és az újonnan létrejött $P'$ populációkat,
és rendezzük az egyedeket \aref{eqn:SOCNDP_GA_FITNESZ_FUNCTION} fitnesz függvény alapján.
Növekvő sorrendbe rendezzük őket, mivel nem szabad elfelejtenünk, hogy célunk végső soron a páronkénti konnektivitás minimalizálása.
Ezután kiválasztjuk az első $N$ egyedet, és ezeket visszük tovább a következő iterációba.
Genetikus algoritmusunk szelekciós szakaszát \aref{alg:GENETIKUS-ALGORITMUS:SZELEKCIO} kódrészlet ismerteti.
\begin{algorithm}[h]
  \caption{Selection Operator}\label{alg:GENETIKUS-ALGORITMUS:SZELEKCIO}
  \begin{algorithmic}[1]
    \Function{Selection}{$N, P, P'$}
    \State $P \leftarrow P \cup P'$
    \State \Call{Sort}{$P$}\Comment{Sort individuals by fitness function in ASC order}
    \State \Return $P\left[ {:}\,N \right]$\Comment{Take best $N$ solutions}
    \EndFunction
  \end{algorithmic}
\end{algorithm}

