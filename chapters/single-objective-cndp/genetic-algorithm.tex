\section{Genetikus algoritmus}\label{sec:GENETIKUS_ALGORITMUS}

A genetikus algoritmus a metaheurisztikák osztályába tartozik, és a természetes kiválasztódás inspirálta.
Egy globális optimalizáló, amely gyakran használt optimalizációs és keresési problémák esetében,
ahol a sok lehetséges megoldás közül a legjobbat kell megkeresni.
Azt hogy egy megoldás mennyire jó, a fitness függvény mondja meg.

A genetikus algoritmus mindig egy populációnyi megoldással dolgozik.
A populációba egyedek tartoznak, amelyek egyenként egy-egy megoldásai a feladatnak.
Az algoritmus minden iterációban egy új populációt állít elő az aktuális populációból úgy,
hogy a \textbf{szelekciós operátor} által kiválasztott legrátermettebb szülőkön alkalmazza a
\textbf{rekombinációs} és \textbf{mutációs operátorokat}.

Ezen algoritmusok alapötlete az, hogy minden újabb generáció
az előzőnél valamelyest rátermettebb egyedeket tartalmaz, és így a keresés folyamán
egyre jobb megoldások születnek.
