\section{Mohó algoritmus}\label{sec:MOHO_ALGORITMUS}

\subsection{Általánosan}
Egy mohó algoritmus egy egyszerű és intuitív algoritmus, amely gyakran használt
optimalizációs feladatok megoldására. Az algoritmus helyi optimumok megvalósításával próbálja
megtalálni a globális optimumot.

Habár a mohó algoritmusok jól működnek bizonyos feladatok esetében,
mint pl. Dijkstra-algoritmus, amely egy csomópontból kiindulva meghatározza a legrövidebb utakat,
vagy Huffman-kódolás, amely adattömörítésre szolgál, de sok esetben nem eredményeznek optimális megoldást.
Ez annak köszönhető, hogy míg a mohó algoritmus függhet az előző lépések választásától,
addig a jövőben meghozott döntésektől független.

Az algoritmus minden lépésben mohón választ, folyamatosan lebontva a feladatot kisebb feladattá.
Más szavakkal, a mohó algoritmus soha nem gondolja újra választásait.

\subsection{Saját mohó algoritmus}
A mohó algoritmus kiindul a gráf csúcslefedéséből.
\footnote{
  Angolul: vertex cover.
}
Ez lesz a kezdeti $S$ megoldásunk.
A maradék csomópontok $V \setminus S$  a gráf maximális független csúcshalmazát
\footnote{
  Angolul: maximal independent set.
}
\emph{MIS} alkotják.
Mivel majdnem biztos, hogy a megoldásunkban több, mint k csomópont lesz, ezért mohón elkezdünk kivenni csomópontokat $S$-ből,
majd ezeket hozzáadni \emph{MIS}-hez, amíg $\abs{S} > k$.
A hozzáadott csomópont az lesz, amelyiket ha visszatesszük az eredeti gráfba,
akkor a minimum értéket téríti vissza a páronkénti konnektivitásra a keletkezett gráfban.

Mivel több olyan csomópont lehet, amelyeket ha visszateszünk az eredeti gráfba,
akkor ugyanazt a minimális értéket adják vissza a páronkénti konnektivitásra,
ezért ezeket eltároljuk a B halmazban, és minden lépésben random módon határozzuk meg,
hogy melyik kerüljön vissza \emph{MIS}-be. Ezzel az eljárással garantáljuk,
hogy a mohó algoritmusunk különböző megoldásokat fog adni többszöri futtatások esetén.
A CNDP esetén a mohó algoritmust \aref{alg:GREEDY_ALGORITMUS} kódrészlet szemlélteti.
\begin{algorithm}[t]
  \caption{iGreedy}\label{alg:GREEDY_ALGORITMUS}
  \begin{algorithmic}[1]
    \Require $G, k$
    \State $S \leftarrow \Call{Vertex Cover}{G}$

    \While{$\abs{S} > k$}
    \State $B \leftarrow \argmin_{i \in S} f(S \setminus \left\{ i \right\})$
    \State $S \leftarrow S \setminus \left\{ \Call{Select}{B} \right\}$
    \EndWhile

    \State \Return $S$
  \end{algorithmic}
\end{algorithm}

