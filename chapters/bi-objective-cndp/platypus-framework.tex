\section{A Platypus keretrendszer}\label{sec:PLATYPUS_FRAMEWORK}


A Platypus keretrendszert \citeN{hadka2017platypus} fejlesztette ki.
Ez egy evolúciós számítások elvégzésére szolgáló keretrendszer, amely a többcélú evolúciós algoritmusokra összpontosít.
A keretrendszer a következő elemekből tevődik össze:

\begin{itemize}
  \itemsep-0.17em
  \item[\textbullet] többcélú evolúciós algoritmusokból (NSGAII, NSGAIII, EpsMOEA stb.);
  \item[\textbullet] szelekciós operátorokból (versengő kiválasztás);
  \item[\textbullet] rekombinációs operátorokból (SBX, PCX, HUX stb.);
  \item[\textbullet] mutációs operátorokból (PM, UM, BM stb.);
  \item[\textbullet] reprezentációk különböző típusaiból (valós, bináris, részhalmaz stb.);
  \item[\textbullet] dominancia operátorokból (Pareto-dominancia, $\epsilon$-dominancia, attribútum-dominancia).
\end{itemize}

A Platypus keretrendszer modularizáltságának köszönhetően könnyű új algoritmusokkal vagy operátorokkal kibővíteni a rendszert.
De nem csak kibővíteni lehet, hanem meglévő algoritmusok által használt elemeket lecserélni már meglévő vagy általunk definiált elemekre.
Így az algoritmus váza nem változik, viszont lehetőség nyílik különböző operátorok összehasonlítására egy algoritmus esetén.

A következőkben \aref{sec:DOMINANCIA_OPERATOROK}. részben bemutatott dominancia operátorok implementációt szeretnénk ismertetni a Platypus keretrendszer által biztosított környezetben.

% Nash
\subsection{Operátorok definiálása}


\subsubsection{Nash-dominancia}
\Aref{lst:BOCNDP-NASH-DOMINANCE}. kódrészlet szemlélteti \aref{eqn:NASH_DOMINANCIA} képlettel leírt Nash-dominancia implementációját a Platypus keretrendszer segítségével.

\lstinputlisting[
  language={Python},
  caption={A Nash-dominancia implementációja a Platypus keretrendszerben.},
  label={lst:BOCNDP-NASH-DOMINANCE}
]{./progfiles/bi-objective-cndp/bocndp_nash_dominance.py}

Ahhoz, hogy egy új dominancia operátort vezethessünk be a rendszerbe, amit majd kipróbálhatunk különböző multikritériumú genetikus algoritmusok esetén,
egy osztályt kell létrehoznunk, amely a Platypus keretrendszer által biztosított \emph{Dominance} osztályt örököli.
Ezt az osztályt mi úgy neveztük el, hogy \emph{NashDominance}, és a \emph{Dominance} szülőosztály egyetlen metódusát írja fölül, a \emph{compare} metódust.

A \emph{compare} eljárás fog két megoldást - $x$ és $y$ - összehasonlítani, és visszatéríti, hogy melyik Nash-dominálja melyiket, vagy hogy egyik sem dominálja a másikat.
A metódus \aref{lst:NO-CONNECTED-COMPONENTS}. és \aref{lst:CARDINALITY-VARIANCE-COMPONENTS}. kódrészletekkel szemléltetett függvényeket használja:
az első a gráf összefüggő komponenseit számolja ki, a második pedig az összefüggő komponensek számosságának a varianciáját (lásd \aref{eqn:CARDINALITY_VARIANCE_COMPONENTS} egyenletet).
Ezeket a célfüggvényeket akarjuk optimalizálni: az elsőt maximalizálni, míg a másodikat minimalizálni.


\subsubsection{Berge-dominancia}
\Aref{lst:BOCNDP-BERGE-DOMINANCE}. kódrészlet szemlélteti \aref{eqn:BERGE_DOMINANCIA} képlettel leírt Berge-dominancia implementációját a Platypus keretrendszer segítségével.
Az operátor bevezetéséhez a rendszerbe ugyanazon lépéseket kell végrehajtanunk, mint a Nash-dominancia esetén.

\lstinputlisting[
  language={Python},
  caption={A Berge-dominancia implementációja a Platypus keretrendszerben.},
  label={lst:BOCNDP-BERGE-DOMINANCE}
]{./progfiles/bi-objective-cndp/bocndp_berge_dominance.py}


\subsection{Problémák definiálása}
Ahhoz, hogy egy bármilyen problémát meg tudhassunk oldani a Platypus keretrendszer által biztosított algoritmusokkal, szükséges először definiálnunk a feladatot.
Ebben nyújt nekünk segítséget a Platypus \emph{Problem} osztálya, amely két paramétert vár el tőlünk: a döntési változók és a célok számát.
Meg kell adjuk minden döntési változó típusát vagy reprezentációját, valamint minden cél optimalizálási irányát (minimalizálás vagy maximalizálás).
Továbbá felül kell írjuk a \emph{Problem} szülőosztály \emph{evaluate} függvényét, amely egy megoldás kiértékeléséért felelős.
A BOCNDP definiálását hívatott szemléltetni \aref{lst:BOCNDP-PROBLEM}. kódrészlet.

\lstinputlisting[
  language={Python},
  caption={A BOCNDP definiálása a Platypus keretrendszerben.},
  label={lst:BOCNDP-PROBLEM}
]{./progfiles/bi-objective-cndp/bocndp_problem.py}

Amint látható, létrehozunk egy \emph{BOCNDP} nevű osztályt, amely a Platypus által biztosított \emph{Problem} osztály leszármazottja.
A konstruktorban meghívjuk a szülőosztály \emph{init} metódusát, amely beállítja a döntési változók számát $1$-re, a célok számát pedig $2$-re.
Megmondjuk, hogy a döntési változót a $G$ gráf csomóponthalmazának egy $k$ elemű részhalmazával fogjuk ábrázolni, valamint azt, hogy az első célfüggvényt maximalizálni, míg a másodikat minimalizálni szeretnénk.

Ezután felülírjuk a \emph{Problem} osztály \emph{evaluate} metódusát. Itt történik egy megoldás tényleges kiértékelése.
Beállítjuk mint első célfüggvény \aref{lst:NO-CONNECTED-COMPONENTS}. kódrészlettel leírt függvényt, ami a $G$ gráf összefüggő komponenseinek a számát téríti,
\aref{lst:CARDINALITY-VARIANCE-COMPONENTS}. kódrészlettel leírt függvényt pedig, ami az összefüggő komponensek számosságának a varianciáját téríti, mint második célfüggvény.
Ahogy az imént említettük, az első célfüggvényt maximalizálni, míg a másodikat minimalizálni fogjuk.


\subsection{Algoritmusok definiálása}
A Platypus keretrendszer számos kész multikritériumú genetikus algoritmust tartalmaz, de lehetőséget ad új algoritmusok definiálására is az \emph{Algorithm} osztály kiterjesztése révén.
Ugyanakkor lehetővé teszi a már meglévő algoritmusok testreszabását: egy algoritmus példányosításakor vagy felépítésekor megadhatjuk argumentumként a használni kívánt operátorokat.

\lstinputlisting[
  language={Python},
  caption={A BOCNDP megoldása az NSGAII algoritmus és a Nash-dominancia használatával.},
  label={lst:BOCNDP-EXAMPLE}
]{./progfiles/bi-objective-cndp/bocndp_example.py}

\Aref{lst:BOCNDP-EXAMPLE}. kódrészlet szemlélteti az NSGAII algoritmus egy testreszabott változatának a létrehozási folyamatát,
ahol beállítjuk, hogy \aref{lst:BOCNDP-PROBLEM}. kódrészlettel leírt \emph{BOCNDP} problémát oldja meg, és \aref{lst:BOCNDP-NASH-DOMINANCE}. kódrészlettel leírt \emph{NashDominance} osztályt használja mint dominancia operátor.
Ezután lefuttatjuk az algoritmust $10\,000$-es iterációszámmal, utána pedig kiíratjuk a Nash-optimális megoldásokat.


\subsection{Generátorok definiálása}
A Platypus keretrendszer lehetővé teszi, hogy saját generátorokat definiáljunk, amelyek segítségével tetszőlegesen inicializálni tudjuk a kezdeti populációt.
A Platypus által biztosított \emph{Generator} osztály kiterjesztésével, és a \emph{generate} metódus felülírásával érhetjük el ezt.
Három sajátos inicializálási módszert használtunk a BOCNDP esetén, amelyeket a következő részben fogunk részletesen bemutatni.
