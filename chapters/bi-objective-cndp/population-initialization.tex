\section{A kezdeti populáció inicializálása}


Ugyanúgy, mint \aref{ch:EGYCELU_CNDP}. fejezetben bemutatott CNDP esetén, ahol \aref{sec:MEMETIKUS_ALGORITMUS}. részben leírt memetikus algoritmus keretein belül a populációt intelligens módon inicializáltuk, a BOCNDP esetén is kísérletezni fogunk különböző inicializálási módszerekkel.
Ez azért lényeges, mert ha eleve olyan megoldásokból indulunk ki, amelyek magukban tárolnak valamit a hálózat szerkezetéről, akkor megtörténhet, hogy jobb eredményekhez jutunk rövidebb idő alatt.
Ezeket a módszereket szeretnénk a következőkben ismertetni.


\subsection{DFS alapján}
Az első módszer abban áll, hogy kiindítunk egy mélységi bejárást (DFS) a $G$ gráf egy véletlen csomópontjából,
majd az így kapott csomóponthalmaz minden $x$-edik elemét különválasztjuk, $x = \dfrac{\abs{V}}{k}$.
A különválasztott csomópontok halmaza fogja alkotni a kezdeti populáció egy egyedét.
\Aref{alg:DFS-SMART-INIT} algoritmus szemlélteti a mélységi bejáráson alapuló intelligens inicializálási módszert.
\begin{algorithm}[h]
  \caption{Depth-first search solution generator}\label{alg:DFS-SMART-INIT}
  \begin{algorithmic}[1]
    \Function{Dfs Generator}{$G, k, x$}
    \State $start \leftarrow \Call{Select}{V}$
    \State $S \leftarrow \Call{Dfs}{G, start}$
    \State \Return $S\left[ {::}\,x \right]$\Comment{Take every $x$th element}
    \EndFunction
  \end{algorithmic}
\end{algorithm}



\subsection{Fokszám alapján}
A második módszer a csomópontok fokszám központiságán alapszik, vagyis minél több éllel rendelkezik egy csomópont, annál nagyobb valószínűséggel fog bekerülni a kezdeti populáció egy egyedének a halmazába.
A generált megoldás első $x$ elemét a gráf legmagasabb fokszámmal rendelkező csomópontjai képezik, a maradék $(k - x)$ elemét pedig véletlenszerűen kiválasztott csomópontok.
Ha az egyedek halmazait kizárólag a legmagasabb fokszámú csomópontokból építenénk fel, akkor a kezdeti populáció egyedei mind egyformák lennének, hiszen a bemeneti $G$ gráf nem változik.
Ezért szükséges, hogy a generált megoldás egy részét a legnagyobb fokszámú csomópontok, míg a másik részét véletlenszerűen kiválasztott csomópontok alkossák.
\Aref{alg:DEGREE-SMART-INIT} algoritmus szemlélteti a fokszám központiságon alapuló intelligens inicializálási módszert.
\begin{algorithm}[t]
  \caption{Degree solution generator}\label{alg:DEGREE-SMART-INIT}
  \begin{algorithmic}[1]
    \Require $G, k, x$
    \State $V' \leftarrow \Call{Sorted}{V}$\Comment{Sort nodes according to their degree in DESC order}
    \State $S \leftarrow V'\left[ {:}\,x \right]$\Comment{Take first $x$ nodes with the highest degree}

    \While{$\abs{S} < k$}
    \State $node \leftarrow \Call{Select}{V'}$
    \If{$node \notin S$}
    \State $S \leftarrow S \cup \left\{ node \right\}$
    \EndIf
    \EndWhile

    \State $\Call{Shuffle}{S}$
    \State \Return $S$
  \end{algorithmic}
\end{algorithm}



\subsection{Véletlen séta alapján}
A harmadik és egyben utolsó módszer a véletlen sétán alapszik. Elindulunk a $G$ bemeneti gráf egy véletlen módon kiválasztott csomópontjából, és minden lépésben meglátogatjuk a jelenlegi csomópont valamelyik szomszédját, amit ugyancsak véletlen módon választunk ki.
Miközben sétálunk, számon tartjuk mindegyik csomópont esetén, hogy hányszor látogattuk meg.
Ez kulcsfontosságú, mivel ennek alapján fogjuk eldönteni, hogy mely csomópontok kerüljenek be a generált megoldásba.
Minél többször volt egy csomópont meglátogatva, annál nagyobb eséllyel fog bekerülni a kezdeti populáció egy egyedének a halmazába.
A séta hosszát a $t$ változó mondja meg, és lesz egy $p_r$ valószínűség, ami az újrakezdés valószínűségét fogja jelenteni.
Minden lépésben eldöntjük, hogy folytatjuk vagy újrakezdjük a sétát.
Újrakezdés esetén visszatérünk ahhoz a csomóponthoz, amelyikből a sétát indítottuk.
Ha a séta végére nem sikerült $k$ különböző csomópontot meglátogatnunk, akkor a sétát újra indítjuk, de most már egy új csomópontból.
\Aref{alg:RANDOM-WALK-SMART-INIT} algoritmus szemlélteti a véletlen sétán alapuló intelligens inicializálási módszert.
\begin{algorithm}
  \caption{Random walk solution generator}\label{alg:RANDOM-WALK-SMART-INIT}
  \begin{algorithmic}[1]
    \Require $G, k, t, p_r$
    \State $visited \leftarrow \emptyset$

    \While{$True$}
    \State $core \leftarrow \Call{Select}{V}$
    \State $current \leftarrow core$

    \For{$i \leftarrow 1, t+1$}
    \If{$current \in visited$}
    \State $visited\left[current\right] \leftarrow visited\left[current\right] + 1$
    \Else
    \State $visited\left[current\right] \leftarrow 1$
    \EndIf

    \State $restart \leftarrow \Call{Rand Int}{1, 100}$
    \If{$restart \leq p_r$}
    \State $current \leftarrow core$
    \Else
    \State $neighbors \leftarrow \Call{Neighbors}{G, current}$\Comment{Neighbors of the $current$ node}
    \State $current \leftarrow \Call{Select}{neighbors}$
    \EndIf
    \EndFor

    \If{$\abs{visited} \geq k$}
    \State \textbf{break}
    \Else
    \State $visited \leftarrow \emptyset$
    \EndIf
    \EndWhile

    \State $\Call{Sort}{visited}$\Comment{Sort nodes in $visited$ according to visits paid in DESC order}
    \State \Return $visited\left[ {:}\,k \right]$\Comment{Take the first $k$ most visited nodes}
  \end{algorithmic}
\end{algorithm}

